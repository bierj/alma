\section{Discussion}\label{discussion}

An upper bound nucleus size of approximately 10.2 km for comet C/2023 A3 (Tsuchinshan-Atlas) places it within typical ranges observed for
long-period comets according to Meech et al. (2004) albeit a bit on the low end of the spectrum. According to their analysis of long-period comets, comet nuclei ranged from \textless{4} to 56 km in size. Comet Hale-Bopp, a notably large comet long-period Oort cloud comet, had a nucleus diameter around 40--60 km. What is notably anomalous is the range of calculated dust masses. Our value of
\(5.2 - 8.4 \times 10^{7}\) seems too high by one or two orders of magnitude based on dust production observations and simulations done by
Moreno et al. (2025). However, this range of masses is still below measurements of large comet dust masses such as comet Hale-Bopp, which
had an estimated dust mass of \(4.7 \pm 1.1 - 5.3 \pm 1.2 \times 10^{11}\) kg when it was within 1 AU according to Jewitt \& Matthews (1999). One likely source of error is an improper dust opacity. Future work will be needed to determine appropriate estimates for grain size, ice content, etc.\ to find or model more appropriate dust opacities.

Another source of error could be the spectral index from Lelouch et al. (2022). Lelouch et al. fit a spectral power function in the 260 GHz range, yet our data is in the 460 GHz range.

We did not detect any molecular spectral lines for methanol or sulfur monoxide. Both Tang et al. (2024) and Cambianica et al. (2025) measured a depleted comet. Cambianica et al. (2025) were able to detect CN molecular lines in the visual spectrum using the DOLORES spectrograph and derive a production rate, but they detected no other species. They also measured a large dust opacity, in line with observations by Tang et al. (2024), and also in line with our calculations of a rather massive dust coma, which may serve to obscure any molecular lines in the visible spectrum. Also, our detection of the comet continuum was only to \textasciitilde{}\(6\sigma\) confidence compared to \textgreater\(10\sigma\) for Lelouch et al. (2022). Nevertheless, we intend to calculate upper limits for CH\textsubscript{3}OH, SO, and NH\textsubscript{2}D abundances and production rates using the SUBLIMED model (Cordiner 2022). Since we lacked detection of these lines, we can calculate an upper limit for production rates. If production rates exceed had exceed these modeled upper limits, then we should have detected them.

Unfortunately, this is our only data set for A3. The blind pointing required with an instrument such as ALMA can cause some observations to be missed or not centered in the instruments field of view. Comet A3 is an excellent example of the collaboration required to properly
characterize an Oort cloud comet. We'll never have another look at it, so the more eyes and instruments and analyses we have for A3, the better chance we have to properly understand sporadic, transient objects like Oort Cloud comets.
