\section{Introduction} \label{intro}

Comets are important to understanding the formation of the early solar system. They are time capsules containing ices of various chemicals from simple molecules like H2O and CO, to more complex molecules such as methanol and formaldehyde. And the comets which have been preserved the best are the comets which were cast out into the Oort cloud by interactions with the heavier gas and ice giants. At a distance ranging from 3000 AU to 20,000 AU for the inner Oort cloud and 20,000 AU to 100,000 AU for the outer Oort cloud, these comets preserving unaltered materials from the era when our solar system first emerged from a protoplanetary disk. Comet C/2023 A3 (Tsuchinshan-Atlas) (hereafter referred to as A3) is one such comet.

Comet C/2023 A3 (Tsuchinshan-Atlas), discovered independently by the Tsuchinshan Observatory and the ATLAS survey, has attracted considerable interest due to its anticipated brightness and favorable observing conditions during its closest approach to Earth (Tang 2024). Projected to reach a perihelion of 0.391 AU, this comet offers a rare opportunity to study an Oort cloud comet in unprecedented detail \citep{Tang2024}. Based on current calculations\footnote{Eccentricity \textgreater{1} based on ephemeris from JPL Horizons current as of May 7, 2025.}, comet A3 has been ejected from our solar system and will never be seen again. Collecting as many observations as possible with multiple modalities is critical due to the unpredictability and transient nature of Oort cloud comets.

This study presents observational analyses of the recently discovered comet C/2023 A3 (Tsuchinshan-Atlas), employing high-resolution radio
interferometry data from the Atacama Large Millimeter/submillimeter Array (hereafter referred to as ALMA) \citep{ALMA}, including observations from both its 12-meter main array
and the Atacama Compact Array (ACA). ALMA's unparalleled sensitivity and spatial resolution at millimeter and submillimeter wavelengths make it uniquely suited to investigate the composition and structure of cometary nuclei and their surrounding comae.

Furthermore, through observations of comet A3, we aim to characterize upper limits for molecular abundances, dust mass, grain size, and nucleus diameter, thus providing direct constraints on the comet's chemical and physical properties. These data are crucial for enhancing our understanding of cometary activity, refining models of comet formation and evolution, and offering a broader perspective on the diversity and origins of primitive material within the solar system. Ultimately, insights gained from C/2023 A3 will contribute significantly to our knowledge of the pristine conditions prevalent in the outermost reaches of our planetary system now and in the distant past.
