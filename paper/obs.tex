\section{Observations and Data Reduction}

Observations were collected on October 1, 2025, between 14:31 and 16:10 UTC on A3 and three calibration targets, quasars J1058+0133, J1118-1232 and J1130-1449. Data was collected using ALMA Band 8 over 4 spectral windows centered at 459.127 GHZ, 461.079 GHZ, 470.383 GHZ, and 472.127 GHZ with 128, 960, 960, and 3840 channels and 15,625 kHz, 244.141 kHz, 244.141 kHz, and 488.281 kHz channel widths respectively for a total of 2 GHZ, 234.375 MHz, 234.375 MHz, and 1.875 GHz total bandwidth respectively. The 12-meter configuration utilized 47 antennas while the ACA configuration utilized 14 antennas. This yielded, for the 12-meter configuration, baselines from \textasciitilde25-400 meters. Comet A3 was at a heliocentric distance of approximately 0.406 AU moving away from the Sun at an apparent speed of roughly 12.4 km/s. It was at a geocentric distance of approximately 0.74 AU with an apparent speed of 71 km/s towards the telescope. \jjb{Probably should include more details on the data collection specs from the proposal.}

\jjb{Should I mention the observations we didn't analyze? Or should I save that for the discussion section?}

The data was delivered as two measurement sets processed through the CASA Pipeline. To begin, the measurements sets were split to separate the quasar calibrators from the comet. Then those filtered measurement sets were processed using the \emph{\textbf{tclean}} task in CASA \citep{McMullin2007}. Spectral windows 0 and 3 were chosen to analyze the continuum due to their high channel count and broad bandwidth. Once images were generated, they were cleaned by removing the background noise. Images of continuum for the 12m and ACA were opened using the CASA \emph{\textbf{imview}} command. The comet, which was detected near the bottom, right edge of the image, was demarcated by an elliptical region by hand. Then the rms value of the region of the image not contained in the comet region was measured and subtracted from the background to generate a clean continuum image. For improved visualization, the comet was centered using the \emph{\textbf{fixplanet}} and \emph{\textbf{phaseshift}} CASA commands.

\jjb{Not sure how to get figures to appear in observation section.}

\begin{figure*}
    \gridline{\fig{A3.12m.460-471GHZ.cont.png}{0.45\textwidth}{(A)}
        \fig{A3.ACA.460-471GHZ.cont.png}{0.45\textwidth}{(B)}}
    \caption{\textbf{(A).} Continuum from 460 to 471 GHz for the 12m array. The comet is centered at position 0, 0 and the x and y axes indicate distance from the comet centroid determined by the comet region discussed previously. The x axis indicates distance west of the centroid as positive values and distance east as negative values. Likewise, the y axis indicates distance north of the centroid as positive values and distance south as negative values. The color indicates the average continuum flux per beam in units of mJy. Contours indicate multiples of $3\sigma$ confidence levels. The synthesized beam is shown in the lower, left. The orientation of the comet is indicated in the lower, right. S denotes the direction towards the sun. T denotes the direction of the comet's tail. The sphere indicates how the comet is being illuminated by the sun. \textbf{(B).} Continuum for the ACA array. \jjb{Figures aren't aligning properly and I'm not sure what the black lines are adjacent to the y axis.}
        \label{fig:cont}}
\end{figure*}

Figure 1 shows the continuum from 460 to 471 GHz for the 12m and ACA arrays. The comet is centered at position 0, 0 and the x and y axes indicate distance from the comet centroid determined by the comet region discussed previously. The x axis indicates distance west of the centroid as positive values and distance east as negative values. Likewise, the y axis indicates distance north of the centroid as positive values and distance south as negative values. The color indicates the average continuum flux per beam in units of mJy. Contours indicate multiples of $3\sigma$ confidence levels. The synthesized beam is shown in the lower, left. The orientation of the comet is indicated in the lower, right. S denotes the direction towards the sun. T denotes the direction of the comet's tail. The sphere indicates how the comet is being illuminated by the sun.

\begin{figure*}
    \plotone{dust_lmfit_comparison.png}
    \caption{Results of the channel averaging and binning with a width of 24 meters. Each individual point is the real part of the complex visibility where the error bar is calculated according to the propagation of errors described in section 4.1 of \cite{Nixon2020}. \jjb{Describe the fit lines.} The x axis is baseline distance in meters, and the y axis is flux density measured in janskys. \jjb{The text is too small. Legend should have $\chi_{residual}^2$ values. Should this instead be in results?}
        \label{fig:binned}}
\end{figure*}

Figure 2 shows the results of the channel averaging and binning with a width of 24 meters. Each individual point is the real part of the complex visibility where the error bar is calculated according to the propagation of errors described in section 4.1 of \cite{Nixon2020}. \jjb{Describe the fit lines.} The x axis is baseline distance in meters, and the y axis is flux density measured in janskys.

To facilitate further calculations, the visibilities of the two spectral windows at 459 GHz and 472 GHz were combined. Since visibilities from different frequencies were averaged together, the spectral behavior of continuum emission had to be considered. Since the intensity of blackbody radiation for an object of a given temperature varies with frequency, simply averaging together fluxes would not give a representative measure of the object's thermal emission. Fluxes from different frequencies were converted to a common scale using the spectral index, which assumes that flux as a function of frequency follows a power law: \(F_{1}/F_{2} = {(\upsilon_{1}/\upsilon_{2})}^{x}\), where x is the spectral index, and \(F_{i}\ \)are the fluxes at each frequency \(\upsilon\) (Roth, 2025). The first spectral window centered on 459 GHz was chosen as the common frequency and 1.93 given by Lelouch et al. (2022) was used as the spectral index.
