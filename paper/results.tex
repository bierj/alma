\section{Results}\label{results}

To calculate nucleus size, the individual baseline visibilities were averaged over channel and binned. Then the binned visibilities were fit using least squares with the \textbf{lmfit} Python package. The measured flux was assumed to be constant at the nucleus and follow a \(1/\rho\) curve for the coma, where \(\rho\) is the distance from the nucleus (Roth, 2025). \jjb{Results from dust\textunderscore{model}\textunderscore{lmfit} function go here.} Finally, using a \(3\sigma\) upper bound for the intercept parameter, equation 19 from Delbo \& Harris (2002) based on the NEATM model was rearranged to solve for nucleus diameter, giving an upper limit of 10.2 km.

To calculate dust mass, we chose the flux value at the shortest (binned) baseline. The overall integrated dust continuum flux is best determined by the shortest baselines, which sample the widest angular scales (Roth, 2025). We rearranged Equation 5 from Roth et al. (2023) to solve for dust mass, M. The temperature for the dust grains was calculated using a simple equilibrium law, \(T\left( r_{H} \right) = (277\ K)(1 - A)/\sqrt{r_{H}}\), where \(r_{H}\) is the heliocentric distance measured in AU, and A is the Bond albedo. The Bond albedo was assumed to 0.0212. Dust opacity numbers were assumed to be astronomical silicates and gathered from Table 5 by Boissier et al. (2012). A range of dust opacities was obtained assuming a size index of -3, a grain size range of \(10^{- 4} - 10\) mm, porosity of 50\%, wavelengths of 0.8 and 0.45 mm (459 GHz is \textasciitilde0.65 mm), and ice fractions of 0\%, 22\%, and 48\%. Finally, \(3\sigma\) upper limits of dust mass were calculated to be \(5.2 - 8.4 \times 10^{7}\)kg assuming a flux of \(6.037 \times 10^{- 3}\) Jy and \(\sigma = 2.152 \times 10^{- 3}\).
