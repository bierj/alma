\section{Results}\label{results}

To calculate nucleus size, the individual baseline visibilities, averaged over channel and binned, were fit to a power law \(a*x^{b} + c\). The c parameter of the fit describes the flux of the point source i.e. the nucleus, while the b parameter describes the \(1/r\) dependence on the flux of the coma, where r is the distance from the nucleus (Roth, 2025). Figure 3 shows the results of the channel averaging and binning over three different bin widths, 6m, 9m, and 12m. Each individual point is the real part of the complex visibility where the error bar is calculated according to the propagation of errors described in section 4.1 of Nixon et al. (2020). The orange line represents the power law fit to the real part of the visibilities and the fit parameters are shown in the legend. The x axis is baseline distance in meters, and the y axis is flux density measured in janskys. The various fits at different bin widths show close agreement so the largest one, \(9.92 \times 10^{- 4}\ Jy\), was selected. The error of the intercept parameter was \(5.34 \times 10^{- 7}\). Finally, using a $3\sigma$ upper bound for the intercept parameter, equation 19 from Delbo \& Harris (2002) based on the NEATM model was rearranged to solve for nucleus diameter, giving an upper limit of 10.2 km.

To calculate dust mass, we chose the flux value at the shortest (binned) baseline. The overall integrated dust continuum flux is best determined by the shortest baselines, which sample the widest angular scales (Roth, 2025). We rearranged Equation 5 from Roth et al. (2023) to solve for dust mass, M. The temperature for the dust grains was calculated using a simple equilibrium law, \(T\left( r_{H} \right) = (277\ K)(1 - A)/\sqrt{r_{H}}\), where \(r_{H}\) is the heliocentric distance measured in AU, and A is the Bond albedo. The Bond albedo was assumed to 0.0212. Dust opacity numbers were assumed to be astronomical silicates and gathered from Table 5 by Boissier et al. (2012). A range of dust opacities was obtained assuming a size index of -3, a grain size range of \(10^{- 4} - 10\) mm, porosity of 50\%, wavelengths of 0.8 and 0.45 mm (459 GHz is \textasciitilde0.65 mm), and ice fractions of 0\%, 22\%, and 48\%. Finally, $3\sigma$ upper limits of dust mass were calculated to be \(5.2 - 8.4 \times 10^{7}\)kg assuming a flux of \(6.037 \times 10^{- 3}\) Jy and \(\sigma = 2.152 \times 10^{- 3}\).
