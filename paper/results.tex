\section{Results} \label{results}

To calculate nucleus size, the individual baseline visibilities were averaged over channel and binned. Then the binned visibilities were fit using least squares with the \textbf{lmfit} Python package. The measured flux was assumed to be constant at the nucleus and follow a \(1/\rho\) curve for the coma, where \(\rho\) is distance from the nucleus. Rearranging equation 19 from \cite{Delbo2002} to solve for nucleus size, we estimated the nucleus to be \(4.3 \pm 0.5\) km, which gives a \(3\sigma\) upper bound of 5.6 km. Dust mass was calculated based on a range of grain sizes, porosities and ice fractions consistent with a dynamically new (DN) Oort cloud comet. The dust masses ranged from \((9.8 \pm 3.5)\times 10^7\) kg to \((1.1 \pm 0.4)\times 10^8\) kg.

To calculate dust mass, we chose the flux value at the shortest (binned) baseline. The overall integrated dust continuum flux is best determined by the shortest baselines, which sample the widest angular scales. We rearranged Equation 5 from \cite{Roth2023} to solve for dust mass, M. The temperature for the dust grains was calculated using a simple equilibrium law, \(T\left( r_{H} \right) = (277\ K)(1 - A)/\sqrt{r_{H}}\), where \(r_{H}\) is the heliocentric distance measured in AU, and A is the Bond albedo. The Bond albedo was assumed to 0.0212. Dust opacity numbers were assumed to be astronomical silicates and gathered from Table 5 by \cite{Boissier2012}. A range of dust opacities was obtained assuming a size index of -3, a grain size range of \(10^{- 4} - 10\) mm, porosity of 50\%, wavelengths of 0.8 and 0.45 mm (459 GHz \(\approx0.65\) mm), and ice fractions of 0\%, 22\%, and 48\%. Finally, \(3\sigma\) upper limits of dust mass were calculated to be \(5.2 - 8.4 \times 10^{7}\) kg assuming a flux of \(6.037 \times 10^{- 3}\) Jy and \(\sigma = 2.152 \times 10^{- 3}\) kg.

\begin{deluxetable*}{c c c c c c c}
    \tablenum{1}
    \tablecaption{Dust Mass}
    \tablehead{
        \colhead{Wavelength} & \colhead{Size Index} & \colhead{Maximum Particle Size} & \colhead{Porosity} & \colhead{Ice Fraction} & \colhead{Opacity} &  \colhead{Dust Mass} \\
        \colhead{(mm)} & \colhead{} & \colhead{mm} & \colhead{(\%)} & \colhead{\%} & \colhead{(\(m^{-2} kg\))} & \colhead{(kg)}
    }
    \startdata
    0.45 & -3 & 10 & 50 & 0 & 0.274 & \((9.7 \pm 3.4)\times 10^7\) \\
    0.45 & -3 & 10 & 50 & 22 & 0.300 & \((8.8 \pm 3.1)\times 10^7\) \\
    0.45 & -3 & 10 & 50 & 48 & 0.323 & \((8.2 \pm 2.9)\times 10^7\) \\
    \textbf{0.65} & -3 & 10 & 50 & 0 & \textbf{0.245} & \(\mathbf{(1.1 \pm 0.4)\times 10^8}\) \\
    \textbf{0.65} & -3 & 10 & 50 & 22 & \textbf{0.260} & \(\mathbf{(1.0 \pm 0.4)\times 10^8}\) \\
    \textbf{0.65} & -3 & 10 & 50 & 48 & \textbf{0.271} & \(\mathbf{(9.8 \pm 3.5)\times 10^7}\) \\
    0.80 & -3 & 10 & 50 & 0 & 0.206 & \((1.3 \pm 0.5)\times 10^8\) \\
    0.80 & -3 & 10 & 50 & 22 & 0.205 & \((1.3 \pm 0.5)\times 10^8\) \\
    0.80 & -3 & 10 & 50 & 48 & 0.199 & \((1.3 \pm 0.5)\times 10^8\) \\
    \enddata
    \tablecomments{Opacities from Table 5 of \cite{Boissier2012}. Linear interpolation was employed to calculate opacities at 0.65 mm.}
\end{deluxetable*}

We detected no spectral emission lines for \methanol{} or SO. \(3\sigma\) upper limits on abundances relative to H\subs{2}O were calculated using the SUBLIME model by adjusting abundances until the simulated flux matched the \(3\sigma\) RMS noise over line free channels \citep{SUBLIME}. Abundance upper limits relative to H2O for \methanol{} \((J = 7, K = 2-1)\) was 0.038, for \methanol{} \((J = 9-8, K = 2-1)\) was 0.014, for \methanol{} \((J = 10, K = 2-1)\) was 0.017, and for SO \((N = 10-9, J = 11-10)\) was 0.0052.
