\begin{abstract}

    Our solar system is approximately 4.5 billion years old. Comets are remnants of its formation, serving as time capsules to give us an understanding of its natal heritage. Revealing our own solar system's history allows us to gain insights into other young stellar systems and the potential for life elsewhere. C/2023 A3 (Tsuchinshan--ATLAS) was a comet from the Oort cloud on its first and possibly only journey to the inner solar system. The comet grew in brightness to a visual magnitude of -4.9 on October 9, 2024, making it temporarily brighter than Venus and one of the brightest comets of the past century. Spectroscopic observations of comet A3 were conducted on October 1, 2024, using the Atacama Large Millimeter Array (ALMA) radio telescope located in the Atacama Desert in Chile. The Band 8 receiver centered near 460 GHz was used to sample thermal continuum emission from the nucleus and dust in the coma, as well as spectral line emission from \methanol{} (methanol), SO (sulfur monoxide), and NH\subs{2}D (ammonia). From analysis on the continuum, the nucleus size was estimated to be \(4.3 \pm 0.5\) km. Dust mass was calculated based on a range of grain sizes, porosities and ice fractions consistent with a dynamically new (DN) Oort cloud comet. The dust masses ranged from \((9.8 \pm 3.5)\times 10^7\) kg to \((1.1 \pm 0.4)\times 10^8\) kg. Spectral line emissions were not detected. We calculated \(3\sigma\) upper bounds on molecular abundances for \methanol{} and SO, which are in agreement with observations that comet A3 was depleted.

\end{abstract}
