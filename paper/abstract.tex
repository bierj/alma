\begin{abstract}

    Our solar system is approximately 4.5 billion years old. Comets are remnants of its formation, serving as time capsules to give us an understanding of its natal heritage. Revealing our own solar system's history allows us to gain insights into other young stellar systems and the potential for life elsewhere. C/2023 A3 (Tsuchinshan--ATLAS) was a comet from the Oort cloud on its first and possibly only journey to the inner solar system. The comet grew in brightness to a visual magnitude of -4.9 on October 9, 2024, making it temporarily brighter than Venus and one of the brightest comets of the past century. Spectroscopic observations of comet A3 were conducted on October 1, 2024, using the Atacama Large Millimeter Array (ALMA) radio telescope located in the Atacama Desert in Chile. The Band 8 receiver centered near 460 GHz was used to sample thermal continuum emission from the nucleus and dust in the coma, as well as spectral line emission from CH\textsubscript{3}OH (methanol), SO (sulfur monoxide), and NH\textsubscript{2}D (ammonia). Continuum emissions were detected, but spectral line emissions were not detected. Continuum properties of the dust coma were modeled following the methods of \citep{Lellouch2022}. Calculated dust mass and upper limits on the size of the nucleus will be discussed. Future work will entail comparing upper limits on molecular abundances in A3 against other compositional studies that determined it to be depleted in carbon-chain molecules.

\end{abstract}
